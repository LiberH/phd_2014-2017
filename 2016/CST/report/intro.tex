\section*{Introduction}

  %% plan de thèse
  %% définir la cohérence de la thèse
  %% quantifier les efforts à faire pour la dernière année de thèse

  Ce rapport offre un aperçu du travail réalisé durant cette année
  universitaire. Il rend également compte des activités d'enseignements et de
  formations annexes au travail de recherche.

  Le travail réalisé à ce jour consiste en l'implémentation fonctionnelle d'un
  outil de \textit{program slicing} de fichiers binaires exécutables. Cet outil
  permet la génération de modèles de programmes en vue d'en réaliser l'analyse
  temporelle par vérification de modèles temporisés. Les modèles produits à
  partir de fichiers binaires exécutables représentes les graphes de flots de
  contrôles partiellement astraits de leur programme source respectifs.
  L'abstraction de ces graphes de flots de contrôles est obtenue par une
  utilisation astucieuse du \textit{program slicing}.

  Ce rapport est organisé comme suit. La section \ref{sec:work} discute du
  travail réalisé durant cette année universitaire à savoir l'implémentation
  d'un outil de \emph{program slicing}. La section \ref{sec:future} évoque les
  perspectives envisagées pour développer une nouvelle approche d'analyse
  temporelle par vérification de modèles temporisés utilisant l'outil produit.
  La section \ref{sec:other} traite de mes activités doctorales annexes.
  L'annexe \ref{sec:wcet16} est un article qui a été soumis au \emph{workshop}
  WCET 2016.
