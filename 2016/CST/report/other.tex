\section{Activités annexes}
\label{sec:other}
  \subsection{Enseignement}

    Durant l'année universitaire courante j'ai pu effectuer un peu plus de 77
    heures d'encadrement en vacation auprés d'étudiants en première et deuxième
    année de DUT Informatique -- voir ci-dessous pour le détail des
    enseignements dispensés. Cela représente près de 64 heures équivalent TD, le
    nombre maximal d'heures d'encadrement autorisées par an. Soit au total, en
    les cumulant aux heures d'encadrement effectuées l’année universitaire
    précédente, 114 heures de d'encadrement en vacation.
    
    \paragraph{<< Administration Système et Réseau >>}
    { \begin{itemize}
        \item Du 25 Janvier au 21 Mars ;
        \item 14 TD de 1h20 chacun ;
        \item Cours de DUT Informatique deuxième année ;
        \item Déploiement et sécurité des infrastructures réseaux. 
      \end{itemize} }

    \paragraph{<< Architecture et Programmation >>}
    { \begin{itemize}
        \item Du 29 Janvier au 25 Mars ;
        \item 5 TD et 17 TP de 1h20 chacun ;
        \item Cours de DUT Informatique première année ;
        \item Jeu d'instruction IA-32 et \emph{reverse engineering}.
      \end{itemize} }

    \paragraph{<< Architecture des Réseaux >>}
    { \begin{itemize}
        \item Du 28 Avril au 16 Juin ;
        \item 7 TD et 14 TP de 1h20 chacun ;
        \item Cours de DUT Informatique première année ;
        \item Modèle OSI et pile TCP/IP, fonctionnement des couches réseaux
          basses.
      \end{itemize} }

  \subsection{Formation doctorale}

    Au cour des 3 ans de doctorat, un minimum 100 heures de formation doivent
    être validées. Ces heures doivent être réparties équitablement entre
    formation professionelle et formation scientifique.
    
    Durant l'année universitaire courante j'ai pu valider 61 heures de formation
    doctorales dont 36 heures de formation professionnelle et 25 heures de
    formation scientifique -- voir ci-dessous pour le détail des formations
    suivies. Soit au total, en les cumulant aux heures de formations validées
    l'année universitaire précédente, 53 heures de formation professionnelle et
    40 heures de formation scientifique.

    Au cour de l'année universitaire à venir, je devrais suivre une formation
    scientifique de 10 heures ou plus afin de remplir le contrat des 100 heures
    de formation minimum.
  
    \paragraph{<< École d'été MOVEP 2014 >>}
    { \begin{itemize}
        \item À Nantes, du 7 au 11 juillet 2014 ;
        \item Validant 15 heures de formation scientifique ;
        \item Enseignements pour les jeunes chercheurs sur divers aspects de
          modélisation et vérification des applications.
      \end{itemize} }

    \paragraph{<< Journée des doctorants >>}
    { \begin{itemize}
        \item À Nantes, le 21 avril 2016 ;
        \item Validant 10 heures de formation scientifique ;
        \item Rencontre des doctorants de l'école doctorale STIM inscrits en
          deuxième année.
      \end{itemize} }

    \paragraph{<< Regard Neuf : initiation à la pratique du conseil >>}
    { \begin{itemize}
        \item À Nantes, les 20 novembre, 17 et 18 décembre 2015 ;
        \item Validant 18 heures de formation professionnelle ;
        \item Initiation au conseil en entreprise par équipe pluridisciplinaires
          de huit doctorants sur des problématiques réelles en partenariats avec
          des entreprises locales.
      \end{itemize} }

    \paragraph{<< Research : Methodology and strategy >>}
    { \begin{itemize}
        \item À Nantes, les 10, 12, 18, 24 et 25 novembre 2015 ;
        \item Validant 18 heures de formation professionnelle ;
        \item Information et discussions sur de nombreux aspects de
          la Recherche (en anglais).
      \end{itemize} }
  
