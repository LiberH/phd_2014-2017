% !TEX root =  main.tex
\renewcommand{\baselinestretch}{1.2}
\section{Introduction}
\label{sec:introduction}

  %% The timing analysis of a real-time system must determine if the set of
  %% possible executions of the system respect the timing constraints expressed
  %% in its specifications. In order to realise this analysis, the WCET of the
  %% tasks composing the software part of the system must be known. These values
  %% are generaly impossible to compute whereas it is possible to compute upper
  %% bounds for these.
 
  In the recent years, several works have explored techniques to statically
  estimate the worst-case execution times (WCET) of a program using model
  checking~\cite{Gustavsson2010,DOT10,CB13}. The most important issue
  encountered when using model checking to perform WCET estimation is the
  exponential size of the state space that must be exhaustively explored during
  the analysis~\cite{Wil04}. To fight this problem, state-of-the-art model
  checking tools for dense timed systems such as \textsc{Uppaal}~\cite{LPY97}
  use powerful symbolic algorithms and data structures. It has been shown that
  it allows to deal with small but realistic instances of the WCET
  problem~\cite{Gustavsson2010,DOT10}. It is expected that model checking
  technology will continue to improve in the coming years, widening the range of
  instances that can be solved.

  A different and complementary direction to deal with the explosion of the
  state space consists in abstracting the models of the
  programs~\cite{CB13,Brandner2014} or the models of the hardware
  components~\cite{Cassez15}. The idea is to remove the information which does
  not impact the WCET. This work follows this direction, with a focus on the
  models of programs. In the continuation of prior work~\cite{CB13} we explore
  the use of program slicing~\cite{Wei81} at the level of the binary code to
  abstract the model of the program.

  In this paper we introduce \best, a program slicer for binary code. We
  describe its architecture and implementation. We explain the interface
  between \best\ and \h~\cite{KBB12}, a toolchain built around a Hardware
  Architecture Description Language (HADL). Thanks to this interface, the core
  of \best\ is independent from the target instruction set of the binary code.
  We also use \best\ and the Mälardalen benchmarks to show how to compute
  abstract model of programs and report on the benefits that could be reached
  with this approach.
  
  The paper is organized as follows. In Section~\ref{sec:relatedworks} we give
  an overview of related works. In Section~\ref{sec:modelchecking} we outline an approach
  to the estimation of WCET with model checking. In Section~\ref{sec:slicing} we
  provide a summary of program slicing. In Section~\ref{sec:implementation} we
  describe the implementation of \best\ and its interface with \h. In
  Section~\ref{sec:results} we report on an evaluation of the abstraction approach using
  \best{}. In Section~\ref{sec:conclusion} we conclude
  the paper.

  %% Among the various existing techniques, we are interested on those based on
  %% the timed systems verification theory~\cite{DOT10, CB13}. The time bound is
  %% determined by searching the longest path in a model composed of a model of
  %% the binary executable of a task and a set of models of the hardware
  %% architecture components (processor, pipeline, caches, busses, etc.).
  
  %% In comparison with the other approaches, these techniques allow a less
  %% abstract modelisation of the hardware behaviors, which allow to obtain more
  %% accurate bounds. However, their application stills limited due to the state
  %% space explosion of the model to analyse~\cite{Wil04}.
  
  %% Following~\cite{CB13}, our work aims to contribute to control the size of
  %% the state space by abtracting the binary executable. This abstraction
  %% consist in limiting the model state to the values of registers and memory
  %% words that have an impact on the control flow. The computation of this
  %% abstraction is done in two steps. First, the control flow graph of the
  %% program is reconstructed. Second, this graph is analysed using program
  %% slicing to compute the subset of the relevant registers and memory words.
  %% In the following, we focus on this second step.
