%!TEX root = ./main.tex
\section{Related works and contribution}
\label{sec:relatedworks}

  In the context of static WCET analysis, program slicing has been
  explored~\cite{Sandberg2006,Lokuciejewski2009,CB13}. Program slicing is mostly
  used to accelerate the static analysis of flow
  facts~\cite{Sandberg2006,Lokuciejewski2009}. Our goals are different, as well
  as the slicing technique. In contrast to our work, program slicing is applied
  to structured programs at the source code level (or intermediate code
  level~\cite{Sandberg2006}). Our tool works at the binary code level. As a
  positive side effect, it is independent from both the programming language and
  the compiler. To our best knowledge, there is no established tool for slicing
  non-x86 binary code and \best\ aims at filling this gap.

  Our work is the continuation of previous work by Cassez and
  Béchennec~\cite{CB13}. In this work they propose a prototype tool based on the
  classical dataflow equations approach~\cite{Wei81} that computes slices for
  ARM-based binary code. Unlike that, our tool is independent from the target
  instruction set thanks to its interface with the \h\ toolchain. Furthermore,
  our tool is based on a state-of-the-art graph-based approach~\cite{KJL03}. We
  also provide an evaluation focused on the benefits of the program abstraction
  technique.

  Brandner and Jordan~\cite{Brandner2014} propose a graph pruning technique to
  increase the precision of static WCET estimation. Branches of the Control Flow
  Graph (CFG) are pruned based on the criticality of their basic blocks. The
  criticality is defined as the normalized duration of the longest path passing
  through the block~\cite{Brandner2012}. According to the authors this approach
  is akin to ``program slicing in the time domain''. Based on this pruning
  approach, a refinement based WCET calculation meta-algorithm is proposed. We
  do not address a full WCET analysis in this paper. However, their technique
  could be combined with our approach to improve WCET calculation. Such a
  combination should allow to further abstract the program in order to deal with
  state space explosion.






