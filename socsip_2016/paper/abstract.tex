\begin{abstract}
  Cet article décrit l'implémentation d'un outil de calcul de modèles abstraits 
  de programmes binaires. Ces modèles sont destinés à être fournis en entrée
  d'une chaîne de calcul de borne supérieure de pire temps d'exécution basée la
  théorie de la vérification des systèmes temporisés~\cite{CB13}.
  Le calcul consiste à identifier le sous-ensemble des registres et des mots mémoire
  dont la valeur doit être connue pour déterminer le flot d'exécution du
  programme.
  
  %Le calcul se fait en deux étapes.  D'abord, le graphe de flots de contrôle du
  %programme est reconstruit sous forme d'un automate fini à partir d'un fichier
  %binaire éxécutable. Ensuite, ce modèle est analysé \emph{via} \emph{program
      %slicing} pour calculer le sous-ensemble des registres et adresses mémoires
  %pertinents.
\end{abstract}
