\section{Conclusion}
\label{sec:conclusion}

  % Résumé.

  Cet article décrit le principe de fonctionnement d'un outil de calcul de
  modèles abstraits de codes binaires destinés à être traités par une chaîne de
  calcul de borne sur le pire temps d'exécution des programmes par vérification
  de modèle temporisé.  L'outil utilise le \emph{program slicing} pour calculer
  l'ensemble des variables du programme qui influent sur le flot de contrôle.
  Un premier prototype a été développé et validé.  La prochaine étape
  consistera à caractériser le taux de réduction de la taille de l'espace
  d'état atteint par l'outil.


  % Perspectives associées.

  %Les microprocesseurs multi-c{\oe}urs sont de plus en plus employés dans
  %l'embarqué temps-réel à des fins de performances et de baisse de consommation
  %d'énergie. La réalisation de cet outil fait partie d'un travail de
  %développement d'une approche d'analyse temporelle adaptée à ces plateformes
  %matérielles.

