%% Cet article décrit le travail d'implémentation d'un outil de génération de
%% modèles de programmes. Cette implémentation fait partie d'un travail de
%% développement d'une approche d'analyse temporelle des systèmes embarqués
%% temps-réel par vérification de modèles temporisés.
%% 
%% Pour ce faire, les modèles produits sont générés sous forme d'automates finis
%% représentant les CFG des programmes considérés. Les modèles produits sont issus
%% de reconstructions de CFG à partir de fichiers exécutables. Leur taille est
%% limitée en nombre de n{\oe}ud par \textit{program slicing} des CFG.

%% Les microprocesseurs multi-c{\oe}urs sont de plus en plus employés dans l'embarqué
%% temps-réel à des fins de performances et de baisse de consommation d'énérgie.
%% 
%% À la différence des approches existantes d'analyse temporelle par vérification
%% de modèles, l'approche en développement se veut adaptée aux plateformes
%% matérielles multi-c{\oe}urs.

\section{Perspectives}
\label{sec:future}

% Déjà dit en intro
  %La méthode de génération automatique de modèles de programmes qui a été
  %décrite ci-dessus fait partie d'un travail de développement d'une approche
  %d'analyse temporelle des systèmes embarqués temps-réel par vérification de
  %modèle temporisé de ce point de vue similaire à différentes méthodes
  %existantes \cite{CB13, DOT10}.

  \subsection{Modélisation d'une plateforme multi-c{\oe}ur}

    %% % Inéviatbilité de l'utilisation de plateformes multi-c{\oe}urs dans le futur.
    %% 
    %% L'utilisation futur de plateformes multi-c{\oe}urs dans les systèmes
    %% temps-réel semble inévitable et ce à des fins de performances et de
    %% baisse de consommation d'énérgie.

    % Problèmatique des plateformes multi-c{\oe}urs pour l'analyse temporelle.

    Il est largement considéré aujourd'hui que les multi-c{\oe}urs
    seront massivement employés dans le futur de l'embarqué temps-réel à des
    fins de performances et de baisse de consomation d'énérgie. Le problème qui
    se pose est que de telles plateformes ne fournissent pas forcemment de
    garanties sur les propriétés temps-réel des logiciels s'exécutant dessus.

    Les bus de données partagés font partie des équipements les plus critiques
    de ce point de vue du fait des possibles conflits d'accés entre les
    différents c{\oe}urs. De part leur complexité de fonctionnement, on constate
    qu'ils abaissent significativement la precision des calculs de WCET des
    plateformes multi-c{\oe}urs.

    Les temps d'accès à des données en mémoire sont très variables selon qu'ils
    sont réalisés vers la mémoire centrale ou la mémoire cache. Il est donc
    nécessaire de modéliser fidélement la mémoire cache sans quoi les calculs de
    WCET ne peuvent être précis.

    %% % Incompatibilité avec les analyses temporelles actuelles.
    %% 
    %% Cependant, les analyses temporelles actuelles se limitent le plus souvent au
    %% calcul des pires temps d'exécution des tâches en isolation.
    %%
    %% Il nous parrait nécessaire d'étendre l'analyse temporelle des systèmes
    %% embarqués temps-réel pour tenir compte des comportements spécifiques qui
    %% peuvent advenir lors de l'exécution parellèle de plusieurs tâches.

    \vspace{1em}

    % Perspectives associées.

    L'une des ambitions de ce travail est d'étudier l'intérêt des méthodes
    d'analyse temporelle basées sur la vérification de modèles temporisés pour
    les plateformes matérielles multi-c{\oe}urs. Pour ce faire, un modèle
    temporisé d'une plateforme multi-c{\oe}ur PowerPC sera réalisé.  L'analyse
    temporelle portera sur les systèmes obtenus par produit synchronisé de ce
    modèle et des modèles logiciels engendrés par la méthode décrite ci-dessus.

  \subsection{Spécialisation de la vérification}

    % Problèmatique des plateformes multi-c{\oe}urs pour la vérification de modèle.

    La modélisation d'une plateforme matérielle complexe implique un nombre de
    localités et d'horloges important dans le modèle. De fait, malgrès la
    réduction du modèle logiciel grâce à la technique qui a été décrite ici, la
    vérification risque de ne pas aboutir
    soit par manque d'espace mémoire, soit du fait d'un temps de traitement trop
    important dû à l'explosion de la taille de l'espace d'état à traiter.

    \vspace{1em}

    % Perspectives associés.

    Pour pallier ce problème, les algorithmes et structures de données utilisés
    pour la vérification seront spécialisés. Cette spécialisation exploitera
    des propriétés de monotonie pour couper rapidement des branches lors de
    l'exploration de l'espace d'état.

    Ces résultats seront comparés rigouresement aux temps d'exécution mesurés sur une
    plateforme physique PowerPC multi-c{\oe}urs dont le modèle aura été réalise.
    Parmi les programmes dont les WCET seront calculé on retrouvera les
    programmes du banc d'essai de Mälardalen \cite{GBA10} dont les WCET réels
    sont connus.  
    %La méthode de génération automatique de modèles de programmes
    %qui a été décrite ci-dessus fait partie d'un travail de developpement d'une
    %approche d'analyse temporelle des systèmes embarqués temps-réel par
    %vérification de modèle temporisé de ce point de vue similaire à différentes
    %méthodes existantes \cite{CB13, DOT10}.
