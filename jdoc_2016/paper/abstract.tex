\begin{abstract}
  L'analyse temporelle des systèmes embarqués temps-réel est nécessaire pour
  garantir le respect de l'ensemble de leurs contraintes temporelles. Une telle
  analyse doit déterminer si l'ensemble des exécutions possibles d'un système
  respecte l'ensemble des ses contraintes temporelles. Les différentes analyses
  temporelles existantes s'appuient sur un modèle du programme et un modèle de
  la plateforme matérielle.
  
  Cet article décrit l'implémentation d'un outil de génération de modèles de
  programmes pour l'analyse temporelle par vérification de modèles temporisés.
  La génération de tels modèles de programmes se structure en deux étapes.
  D'abord, le graphe de flots de contrôle du programme est reconstruit sous
  forme d'un automate fini à partir d'un fichier binaire éxécutable. Ensuite, ce
  modèle est spécialisé \emph{via} \emph{program slicing} pour pouvoir en
  réaliser une analyse temporelle efficace.
\end{abstract}

\begin{keywords}
  temps-réel, analyse temporelle, vérification de modèles, espace d'état, program slicing.
\end{keywords}

\begin{collaborations}
  Projet FUI FREENIVI
\end{collaborations}
