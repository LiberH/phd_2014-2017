\section{Conclusion}
\label{sec:conclusion}

  % Résumé.

  Cet article décrit le fonctionnement d'un outil de génération de modèles de
  programmes pour l'analyse temporelle des systèmes embarqués temps-réel par
  vérification de modèles temporisés. Les modèles produits sous forme
  d'automates finis sont générés à partir de fichiers exécutables suivant une
  approche de reconstruction de CFG à base de graphes. Ces modèles sont délestés
  d'un certain nombre d'informations qui ne sont pas nécessaires à l'analyse
  temporelle afin de contenir la taille de l'espace d'état qu'ils engendrent.

  % Perspectives associées.

  Les microprocesseurs multi-c{\oe}urs sont de plus en plus employés dans
  l'embarqué temps-réel à des fins de performances et de baisse de consommation
  d'énergie. La réalisation de cet outil fait partie d'un travail de
  développement d'une approche d'analyse temporelle adaptée à ces plateformes
  matérielles.

