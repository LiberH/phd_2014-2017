\section{Travaux connexes}
\label{sec:related-work}

  % Limitation de la de reconstruction de CFG compiler-aware et
  % annotation-based.

  Certaine méthodes de reconstruction de CFG s'appuient sur une connaissance des
  mécanismes internes des compilateurs. D'une part, cette connaissance n'est
  permise que par un travail fastidieux d'analyse des compilateurs. D'autre
  part, les mécanismes internes des compilateurs ne sont pas toujours librement
  accessibles. D'autres méthodes produisent des CFG imprécis qui doivent être
  annotés par la suite, notamment le nombre d'itérations maximal par boucle. De
  la même façon, ces méthodes dépendent d'un travail fastidieux, non sûr et non
  précis d'annotations.

  % FLP15: "Insight: An Open Binary Analysis Framework".

  L'outil \textsc{INSIGHT}, par exemple, n'est pas sujet à ces limitations
  \cite{FLP15}. Dans un permier temps le binaire est transposé dans une
  représentation intermédiaire sur laquelle est réalisée une analyse
  statique. Cette analyse est réalisée par une exécution symbolique de la
  représentation abstraite dans un domaine abstrait où chaque variable est
  substituée par une formule représentant toutes les valeurs possibles que
  celle-ci peut prendre. Pour ce faire, cet outil se base sur un solveur SMT.

  % The00: "Extracting Safe and Precise Control Flow from Binaries".

  Une autre approche, ascendante, classifie chaque instruction complètement et
  correctement avant de réaliser une analyse. Cette approche définit un CFG
  interprocédural (ICFG) en tant que la conjonction d'un graphe d'appel (CG) et
  de CFG, un par fonction. L'ICFG approximé est construit en deux étapes : un
  ICFG conservatif est construit, une analyse statique à base de propagation de
  constantes est réalisée pour raffiner l'ICFG précédent \cite{The00}.

  % Une approche similaire
  % CB13: "Timing Analysis of Binary Programs with UPPAAL".

  Enfin, une dernière méthode \cite{CB13} permet la génération d'un modèle
  logiciel basé sur la reconstruction automatique de CFG à partir de l'analyse
  exclusive d'un fichier binaire, cette génération est effectuée en deux
  phases. Durant la première le graphe de flot de contrôle de l'exécutable est
  reconstruit par \textit{program slicing} \cite{Wei81, Tip95}. Durant la
  seconde étape il est réduit, afin de contenir l'explosion de l'espace d'état,
  également par \textit{program slicing}. L'implémentation décrite ici s'appuie
  sur cette méthode.

  \vspace{1em}

  % Program Slicing à base d'arithmétique
  % Wei81: "Program Slicing".

  Dans sa définition originale l'algorithme de \textit{program slicing} fait
  usage d'un critère de \textit{slicing} et d'un ensemble d'équations pour
  déterminer quelles instructions font partie d'un \textit{slice} \cite{Wei81}.
  Une autre définition utilise une extension du concept de graphe de dépendence
  du programme pour créer des \textit{slices} intraprocéduraux précis pour
  les programmes structurés formés d'une seule procédure et développe le concept
  de graphe de dépendence du système (SDG) pour produire des \textit{slices}
  interprocéduraux précis tenant compte du contexte d'appel de chaque procédure
  \cite{HRB90}.

  Des techniques similaires ont été inventées pour fonctionner avec des
  programmes ayant des flots de contrôle arbitraires. Pour ce faire le CDG est
  construit à partir d'un CFG augmenté et le graphe de dépendance de données
  correspondant à partir du CFG original \cite{BH93, CF94}.

  Une méthode équivalente mais qui ne modifie pas le CFG ou le SDG a été
  développée \cite{Agr94}. Elle s'appuie sur un arbre des postdominants, créé en
  parallèle de la construction du SDG, ainsi qu'un arbre des successeurs
  lexicaux pour déterminer quel saut ajouter au \textit{slice} final.
  
  %% % KJL03: Interprocedural static slicing of binary executables

  %% Une autre méthode défini un procédé pour construire un CFG interprocédural
  %% \cite{KJL03}. Les algorithmes de construction des CFG, CDG, DDG et PDG y sont
  %% modifiés ou étendus pour tenir compte des spécificités des fichiers
  %% exécutables. Des analyses de dépendance de contrôle et de dépendance données
  %% sont réalisées pour chaque fonction du CFG. Un CDG et un DDG sont construits
  %% suite à ces analyses ainsi qu'un PDG grâce aux CDGs et DDGs. Des
  %% \textit{slices} intraproceduraux peuvent être calculés en utilisant le PDG
  %% ainsi que des \textit{slices} interproceduraux en incorporant le PDG au
  %% SDG.
  
  \vspace{1em}

  % Limitation des approches basées sur l'Abstract Interpretation et l'Integer
  % Linear Programming pour l'analyse temporielle.

  Quant à l'analyse temporelle en elle-même, différentes approches
  existent. Certaines se basent par exemple sur la prédiction des comportements
  matériels par interprétation abstraite et sur l'enumération implicite des
  chemins par optimisation linéaire en nombres entiers \cite{HF04}. Cependant,
  ces techniques définissent des algorithmes \textit{ad hoc} pour tenir compte
  des spécificités des architectures pour lesquelles l'analyse temporelle se
  porte. Un changement d'architecture implique alors l'écriture d'un nouvel
  algorithme. De plus les résultats actuellement obtenus avec ces méthodes
  surestiment significativement les bornes supérieures des temps d'exécutions
  réel. Cela peut s'expliquer par le fait que les modèles des comportements des
  équipements matériels qui y sont manipulés sont très abstraits.
