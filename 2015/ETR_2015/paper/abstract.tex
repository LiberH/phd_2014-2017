\begin{abstract}
  L'analyse temporelle des systèmes embarqués temps-réel est nécessaire pour
  garantir le respect de l'ensemble de leurs contraintes temporelles. Une telle
  analyse doit déterminer si l'ensemble des exécutions possibles d'un système
  respecte l'ensemble des ses contraintes temporelles. Les différentes analyses
  temporelles existantes s'appuient sur un modèle du programme et un modèle de
  la plateforme matérielle.
  
  Cet article décrit l'implémentation d'un outil de génération de modèles de
  programmes pour l'analyse temporelle par vérification de modèles
  temporisés. Le processus de génération est composé de deux étapes. La première
  réalise une reconstruction du graphe de flot de contrôle du programme à partir
  d'un fichier éxécutable. La seconde procède à une \textit{simplification} du
  graphe de flot de contrôle par \textit{slicing}.
\end{abstract}
