\documentclass{beamer}
\setbeamertemplate{navigation symbols}{}
\setbeamertemplate{footline}[frame number]
\setbeamertemplate{itemize items}[circle]

\usepackage[utf8]{inputenc}
\usepackage[T1]{fontenc}
\usepackage{lmodern}
\usepackage[frenchb]{babel}
\usepackage{amsmath}
\usepackage{amssymb}
\usepackage{listings}
\usepackage{graphicx}

\title{%
  Séminaire d'équipe \\
  {\bf Notions de \emph{Program Slicing}}}
  %et application aux Programmes Binaires
\author{%
  Armel Mangean \\
  {\small\tt armel.mangean@irccyn.ec-nantes.fr}}
\institute{%
  IRCCyN, équipe Sytème Temps-Réel \\
  École Centrale de Nantes}
\date{Octobre 2015}

\begin{document}

  \begin{frame}
    \titlepage
  \end{frame}

  \begin{frame}
    \frametitle{Plan}
    \tableofcontents
  \end{frame}

  \section{Introduction}
    \subsection{Définitions} % 2 slides
    % Example avec programme en langage haut niveau
      \begin{frame}
        \frametitle{\subsecname}
        
        Soit $\mathcal{LI} = \mathcal{L} \times \mathcal{I}$ un ensemble
        d'instructions étiquetées avec :
        \begin{itemize}
          \item $\mathcal{L}$, un ensemble fini d'étiquettes ;
          \item $\mathcal{I}$, un ensemble fini d'instructions ;
            \item $\nexists (l,i),(l',i') \in \mathcal{L} \times \mathcal{I}$
              tel que $l = l'$.
        \end{itemize}
        \vspace{1em}\pause

        Soit $P$ un sous-ensemble fini de $\mathcal{LI}$, un programme.
        \\ Soit $\mathcal{V}$ un ensemble de variables.
        \\ $\mathcal{V}_{\iota} \subseteq \mathcal{V}$ est l'ensemble des
        variables de $\iota \in \mathcal{LI}$.
        \\ Un état $s$ de $P$ est une valuation de $\mathcal{V}$.
      \end{frame}
      
      \begin{frame}
        \frametitle{\subsecname}
        Soit $\mathcal{S}$ l'ensemble des états de $P$.
        \\ Soit $<\iota>: \mathcal{S} \rightarrow \mathcal{S}$ la
        \emph{sémantique} de $\iota \in \mathcal{LI}$.
        \\ Une exécution de $P$ à partie de l'état initial $s_0$ est une séquence
        \begin{align*}
          run(P,s_0) = s_0 \iota_0 s_1 \iota_1 \cdots s_{n-1} \iota_{n-1} s_n
        \end{align*}
        \vspace{1em}\pause

        
        Une trace $TR(\varrho)$ de l'exécution $\varrho$ est une séquence
        \begin{align*}
          TR(\varrho) = q_0 q_1 \cdots q_n \in (\mathcal{S} \times \mathcal{LI})
        \end{align*}
      \end{frame}

      \begin{frame}
        \frametitle{\subsecname}

        Soit $\mathcal{C} = \langle \iota, V \rangle$ un critère de
        \emph{slicing} avec $\iota \in P$ et $V \subseteq \mathcal{V}$.

        Soit la projection $proj_V: \mathcal{S} \rightarrow \mathcal{S}$
        \begin{align*}
          proj_V(s) = \text{ la valuation }s \text{ restreinte aux variables de }V\text{.}
        \end{align*}
        \vspace{1em}\pause

        Soit la projection $proj: (\mathcal{S} \times \mathcal{LI})
        \rightarrow (\mathcal{S} \times \mathcal{LI}) \cup \{\epsilon\}$ :
        \[ proj_I(s,\iota) =
        \begin{cases}
          \epsilon & \text{ if } \iota \notin I \\
          (proj_{\mathcal{V}_{\iota}}(s),\iota)& \text{ sinon.}
        \end{cases}
        \]
        \vspace{1em}\pause
        
        Soit la projection $proj^*$, l'extension de $proj$ aux traces, telle que :
        \[ proj^*(u) =
        \begin{cases}
          \epsilon                & \text{ if } u = \epsilon \\
          proj^*(w).proj(s,\iota) & \text{ if } u = w.(s,\iota)
        \end{cases}
        \]
      \end{frame}

      \begin{frame}
        \frametitle{\subsecname}
        $S_{\mathcal{C}} \subseteq P$ est un \emph{slice} de $P$ pour le critère
        de \emph{slicing} $\mathcal{C}$ si :
        \begin{align*}
          proj^*(TR(run(P,s_0))) = proj^*(TR(run(S_{\mathcal{C}},s_0))
        \end{align*}
        \vspace{1em}\pause
        
        {\bf $\rightarrow$ On cherche $S_{\mathcal{C}}$ tel que $|S_{\mathcal{C}}|$ soit
          minimale.}
      \end{frame}

      \subsubsection*{Intra- et interprocéduralité}
      \begin{frame}
        \frametitle{\subsecname}
        \framesubtitle{\subsubsecname}

        Le \emph{slicing} intraprocédural consiste au calcul d'un \emph{slice}
        d'une unique procédure monolithique.
        \vspace{1em}\pause
        
        Le \emph{slicing} interprocédural consiste au calcul d'un \emph{slice}
        d'un programme entier où le \emph{slice} dépasse les frontière d'un
        appel de procédure.
        \vspace{1em}\pause

        La principale difficulté du \emph{slicing}
        interprocédural est de prendre en compte correctement le contexte d'appel
        d'un procédure appelée.
      \end{frame}
      
  \subsection{Exemple} % 1 slide
    %\subsection{Weiser, 81}
    % Algorithm:
    % - Intra- and interprocedual
    % - On structured code
    % Limitations:
    % - Not suitable for unstructured and binary code
    % > Seul des techniques basées graphes manipulent des binaires
    \lstset{ %
  backgroundcolor=\color{white},   % choose the background color; you must add \usepackage{color} or \usepackage{xcolor}
  basicstyle=\footnotesize\ttfamily,
  keepspaces=true,
  language=C,
  stepnumber=1,
  tabsize=2,
}

\begin{frame}[fragile]
\frametitle{\subsecname}

\begin{onlyenv}<1>
\begin{figure}
\begin{lstlisting}
L1:  int i = 1;
L2:  int sum = 0;
L3:  int product = 1;
L4:  for (; i < N; ++i)
      {
L5:      sum += i;
L6:      product *= i;
      }
L7:  write(sum);
L8:  write(product);
\end{lstlisting}
\caption{Un programme $P$}
\end{figure}
\end{onlyenv}

\begin{onlyenv}<2>
\begin{figure}
\begin{lstlisting}
L1:  int i = 1;
L2:  int sum = 0;
L3:
L4:  for (; i < N; ++i)
      {
L5:      sum += i;
L6:
      }
L7:  write(sum);
L8:
\end{lstlisting}
\caption{Un slice $S_{\langle L7, \{sum\} \rangle}$}
\end{figure}
\end{onlyenv}
\end{frame}

      
    \subsection{Intérêt} % 2 slides
    % Analyse temporelle basée vérification de modèles
    % Réduction de l'espace d'état (vis-à-vis des registres)
    % Slicing des instructions de contrôle
      \begin{frame}
        \frametitle{\subsecname}

        \begin{itemize}
          \item Débogage, maintenance, optimisation
          \item Génération de test, parallélisation
          \item \emph{Reverse engineering}, \emph{cracking}
          \item \dots
            \vspace{1em}\pause
          \item Estimation de WCET (et BCET)
        \end{itemize}
      \end{frame}

      \begin{frame}
        \frametitle{\subsecname}

        Analyse temporelle par vérification de modèle
        \begin{itemize}
          %\item Domaine d'application : tâches non-isolées
          \item Modularité propice à la modélisation des systèmes concurrents
          \item Modèles matériels \emph{exacts}
          \item Estimations précises de WCET
          %\item Analyse des chemins d'exécution associés au WCET
          \item Synthèse des valeurs d'entrées des WCET estimés
          % Prenant en compte d'autres tâches, des interruptions, du partage des
          % bus et cache, ...
            \vspace{1em}\pause
          \item Explosion de l'espace d'état
        \end{itemize}
      \end{frame}

      \begin{frame}
        \frametitle{\subsecname}

        Program Slicing
        \begin{itemize}
          \item Identification des instructions sans influence sur le
            flot de contrôle
          \item Retrait des registres de ces instructions du modèle
          \item Contention de l'explosion de l'espace d'état
        \end{itemize}
      \end{frame}

      \begin{frame}
        \frametitle{\subsecname}

        \begin{figure}
          \centering
          \includegraphics[scale=.4]{img/schema.png}
          \caption{Approche envisagée}
        \end{figure}        
      \end{frame}
    
  \section{Une approche intraprocedural}
    \begin{frame}
      \frametitle{Plan}
      \tableofcontents[currentsection]
    \end{frame}
    
    %\subsection{Fonctionnement général} % Intraprocedural
    % Schéma:
    % Programme -> CFG -> (PDT -> CDG || DDG) -> PDG
    % + Analyse d'accessibilité
      \subsection{Structures de données}

      \begin{frame}
        \frametitle{\secname}
        \framesubtitle{\subsecname}

        \begin{figure}
          \centering
          \includegraphics[scale=.33]{img/process.png}
          \caption{Structures de données utilisées}
        \end{figure}
      \end{frame}
      
      \subsubsection*{Control Flow Graph}
      % Data structures:
      % - CFG
      \begin{frame}
        \frametitle{\subsecname}
        \framesubtitle{\subsubsecname}
        
        %Un bloc de base est une séquence d'instructions avec :
        %\begin{itemize}
        %  \item Un unique n{\oe}ud d'entrée ;
        %  \item Un unique n{\oe}ud de sortie.
        %\end{itemize}
        %\vspace{1em}
        
        Le CFG du programme $P$ est un tuple $\mathcal{G} = \langle
        \mathcal{N},\mathcal{E},n_0,n_e \rangle$ avec :
        \begin{itemize}
          \item $\mathcal{N}$, l'ensemble fini des instruction étiquetées\\
            (bijection sur $\mathcal{LI}$) ;
          \item $\mathcal{E} \subseteq (\mathcal{N} \times \mathcal{N})$, un
            ensemble fini d'arcs orienté ;
          \item $n_0 \in \mathcal{N}$, le n{\oe}ud d'entrée ;
          \item $n_e \in \mathcal{N}$, le n{\oe}ud de sortie.
        \end{itemize}
      \end{frame}


      \begin{frame}
        \frametitle{\subsecname}
        \framesubtitle{\subsubsecname}
        
        \begin{figure}
          \centering
          \includegraphics[scale=.4]{img/cfg-fix.png}
          \caption{Le CFG de $P$}
        \end{figure}  
      \end{frame}

      
      \subsubsection*{Data Dependence Graph}
      \begin{frame}
        \frametitle{\subsecname}
        \framesubtitle{\subsubsecname}

          Il y a  une dépendence de donnée entre le n{\oe}ud $n$ et le n{\oe}ud $d$ si :
          \begin{itemize}
                \item $n$ utilise une donnée $v$ définie par $d$ ;
                \item $\exists p$ un chemin de $c$ à $n$ sur $\mathcal{G}$ ;
                \item $\nexists m \in p \ \{c,n\}$ tel que $m$ défini $v$.
          \end{itemize}
      \end{frame}
        
      \begin{frame}
        \frametitle{\subsecname}
        \framesubtitle{\subsubsecname}

        \begin{onlyenv}<1>
          \begin{figure}
            \centering
            \includegraphics[scale=.4]{img/ddg_i-fix.png}
            \caption{DDG pour {\tt i}}
          \end{figure}
        \end{onlyenv}

        \begin{onlyenv}<2>
          \begin{figure}
            \centering
            \includegraphics[scale=.4]{img/ddg_sum-fix.png}
            \caption{DDG pou {\tt sum}}
          \end{figure}
        \end{onlyenv}

        \begin{onlyenv}<3>
          \begin{figure}
            \centering
            \includegraphics[scale=.4]{img/ddg_product-fix.png}
            \caption{DDG pour {\tt product}}
          \end{figure}
        \end{onlyenv}
      \end{frame}
      
    \subsubsection*{Post-Dominator Tree}
    %\subsection{Lengauer et al., 79}
    % Data structures:
    % - PDT
      \begin{frame}
        \frametitle{\subsecname}
        \framesubtitle{\subsubsecname}
        
        Soit $\mathcal{G} = \langle \mathcal{N}, \mathcal{E}, n_0, n_e \rangle$
        et $n, d, p \in \mathcal{N}$.
        \vspace{1em}

        Le n{\oe}ud $d$ est un dominateur du n{\oe}ud $n$
        si tous les chemins de $n_0$ au n{\oe}ud $n$
        contiennent le n{\oe}ud $d$. \\
        $\rightarrow$ Par défintion, tout n{\oe}ud se domine lui-même.
        \vspace{1em}\pause

        Soit $\mathcal{E}^{-1} = \{(v, u) | (u, v) \in \mathcal{E}\}$.\\
        Le n{\oe}ud $p$ est un post-dominateur du n{\oe}ud $n$
        si $p$ est dominateur du n{\oe}ud $n$
        dans le graphe $\mathcal{G} = \langle \mathcal{N}, \mathcal{E}^{-1},
        n_e, n_0 \rangle$.
        
        Le n{\oe}ud $s$ est un dominateur strict du n{\oe}ud $n$
        si \\ $s$ domine de $n$ et $s \neq n$. \\
        $\rightarrow$ Par définition, le n{\oe}ud $n_0$ n'a pas de dominateur strict.
        \end{frame}
        
      \begin{frame}
        \frametitle{\subsecname}
        \framesubtitle{\subsubsecname}
        
        Le n{\oe}ud $i$ est un dominateur immédiat du n{\oe}ud $n$
        si $i$ domine strictement $n$
        et $i$ ne domine strictement aucun autre n{\oe}ud $m$
        tel que $m$ domine strictement $n$. \\
        $\rightarrow$ Par définition, tout n{\oe}ud a un unique dominateur immédiat
        sauf $n_0$ qui n'en a aucun.
        \vspace{1em}\pause

        L'arbre $T$ est un arbre de dominateurs
        si chaque n{\oe}ud parent est dominateur immédiat des ses fils. \\
        $\rightarrow$ Par défintion, $n_0$ est la racine d'un tel arbre.
      \end{frame}

      \begin{frame}
        \frametitle{\subsecname}
        \framesubtitle{\subsubsecname}

        \begin{onlyenv}<1>
          \begin{figure}
            \centering
            \includegraphics[scale=.4]{img/rcfg-fix.png}
            \caption{CFG inversé}
          \end{figure}
        \end{onlyenv}
        
        \begin{onlyenv}<2>
          \begin{figure}
            \centering

            % L8 dom {L8-L1} idom {L7   }
            % L7 dom {L7-L1} idom {L4   }
            % L6 dom {L6,L5} idom {L5   }
            % L5 dom {L5   } idom {     }
            % L4 dom {L6-L1} idom {L3,L6}
            % L3 dom {L3-L1} idom {L2   }
            % L2 dom {L2-L1} idom {L1   }
            % L1 dom {L1   } idom {     } 

            \includegraphics[scale=.4]{img/pdt-fix.png}
            \caption{PDT}
          \end{figure}
        \end{onlyenv}
      \end{frame}

    
      \subsubsection*{Control Dependence Graph}
      \begin{frame}
        \frametitle{\subsecname}
        \framesubtitle{\subsubsecname}

        \begin{block}{Control Dependence Graph}
          Il y a une dépendance de contrôle entre le n{\oe}ud $n$ et le n{\oe}ud $c$ si :
          \begin{itemize}
            \item intuitivement, l'exécution de $c$ détermine si $n$ est exécuté.
            \item formellement, 
              \begin{itemize}
                \item $\exists p$ un chemin de $c$ à $n$ sur $\mathcal{G}$ ;
                \item $n$ ne post-domine pas $c$.
              \end{itemize}
          \end{itemize}
        \end{block}
      \end{frame}

      \begin{frame}
        \frametitle{\subsecname}
        \framesubtitle{\subsubsecname}

        \begin{figure}
          \centering
          \includegraphics[scale=.4]{img/cdg-fix.png}
          \caption{CDG}
        \end{figure}
      \end{frame}

    \subsubsection*{Program Dependence Graph}
    %\subsection{Ferrante et al., 87}
    % Data structures:
    % - DDG
    % - CDG
    % - PDG

      \begin{frame}
        \frametitle{\subsecname}
        \framesubtitle{\subsubsecname}

        Le PDG issu du CFG $G$ est un tuple $\mathcal{P} = \langle
        \mathcal{N},\mathcal{E},n_0,n_e \rangle$ avec :
        \begin{itemize}
          \item $\mathcal{N}$, l'ensemble des n{\oe}ud de $\mathcal{G}$ ;
          \item $\mathcal{E}$, l'union des ensembles d'arcs du DDG et du CDG issu de $\mathcal{G}$ ;
          \item $n_0 \in \mathcal{N}$, le n{\oe}ud d'entrée de $\mathcal{G}$ ;
          \item $n_e \in \mathcal{N}$, le n{\oe}ud de sortie de $\mathcal{G}$.
        \end{itemize}
      \end{frame}

      \begin{frame}
        \frametitle{\subsecname}
        \framesubtitle{\subsubsecname}

          \begin{figure}
            \centering
            \includegraphics[scale=.4]{img/pdg-fix.png}
            \caption{PDG}
          \end{figure}
      \end{frame}

    \subsection{Algorithme}
      \begin{frame}
        \frametitle{\subsecname}

        \begin{onlyenv}<1>
          \begin{figure}
            \centering
            \includegraphics[scale=.4]{img/access-fix.png}
            \caption{Analyse d'accessibilité pour $L7$}
          \end{figure}
        \end{onlyenv}
            
        \begin{onlyenv}<2>
          \begin{figure}
            \centering
            \includegraphics[scale=.4]{img/slice-fix.png}
            \caption{Le slice $S_{\langle L7, \{sum\} \rangle}$}
          \end{figure}
        \end{onlyenv}
      \end{frame}

      \appendix
      \addtocounter{part}{-1}

      %% \section{LT79}
      %% \begin{frame}
      %%   \frametitle{\secname}

      %%   \begin{onlyenv}<1>
      %%     \begin{figure}
      %%       \centering
      %%       \includegraphics[scale=.4]{img/rcfg-fix.png}
      %%       \caption{CFG inverse}
      %%     \end{figure}
      %%   \end{onlyenv}

      %%   \begin{onlyenv}<2>
      %%     \begin{figure}
      %%       \centering
      %%       \includegraphics[scale=.4]{img/dfs-fix.png}
      %%       \caption{Arbre couvrant $\mathcal{T}$ issu du \emph{DFS}}
      %%     \end{figure}
      %%   \end{onlyenv}

      %%   \begin{onlyenv}<3>
      %%   $u \xrightarrow{\bullet} v$, $u$ est un ancetre de $v$ dans l'arbre couvrant $\mathcal{T}$. \\
      %%   $u \xrightarrow{+} v$, $u \xrightarrow{\bullet} v$ et $u \neq v$.
      %%     \vspace{1em}
          
      %%   Soit $w \neq n_0$ :
      %%   \begin{align*}
      %%   sdom(w) = \\
      %%     min(  \{&\ v       &|&\ &(&v,w) \in \mathcal{E}                     &\text{ et }& v < w\} \\
      %%     \cup\ \{&\ sdom(u) &|&\ \exists &(&v,w) \in \mathcal{E},\ u \xrightarrow{\bullet} v &\text{ et }& u > w\}
      %%     )
      %%   \end{align*}

      %%   Soit $w \neq n_0$ et $sdom(u)$ minimum pour $sdom(w) \xrightarrow{+} u \xrightarrow{\bullet} v$ :
      %%   \begin{align*}
      %%     idom(w) =
      %%     \begin{cases}
      %%       sdom(w) & \text{si } sdom(w) = sdom(u)\text{,} \\
      %%       idom(u) & \text{sinon.}
      %%     \end{cases}
      %%   \end{align*}
      %%   \end{onlyenv}
            
      %%   \begin{onlyenv}<4>
      %%     \begin{figure}
      %%       \centering
      %%       \includegraphics[scale=.4]{img/sdom-fix.png}
      %%       \caption{Semi-dominateurs}
      %%     \end{figure}
      %%   \end{onlyenv}

      %%   \begin{onlyenv}<5>
      %%     \begin{figure}
      %%       \centering
      %%       \includegraphics[scale=.4]{img/pdt-fix.png}
      %%       \caption{PDT}
      %%     \end{figure}
      %%   \end{onlyenv}
      %% \end{frame}

      %% \begin{frame}
      %%   \frametitle{\secname}
        
      %%   \begin{onlyenv}<1>
      %%     \begin{figure}
      %%       \centering
      %%       \includegraphics[scale=.4]{img/dfs2.png}
      %%       \caption{Autre example}
      %%     \end{figure}
      %%   \end{onlyenv}
      %% \end{frame}


      
  %%   \section{Une approche interprocedural}
  %%   \subsection{Structures de données}
  %%   \subsubsection*{System Dependence Graph}
  %%   % Data structures:
  %%   % - SDG
  %%   % Algorithm:
  %%   % - Intra- and interprocedural
  %%   % - On structured code
  %%     \begin{frame}
  %%       \frametitle{\subsecname}
  %%       \framesubtitle{\subsubsecname}
  %%     \end{frame}

  %%   \subsection{Algorithme}
  %%     \begin{frame}
  %%       \frametitle{\subsecname}
  %%       \framesubtitle{\subsecname}
  %%     \end{frame}

  %% %% \section{Application aux programmes binaires}
  %% %%   \begin{frame}
  %% %%     \frametitle{Plan}
  %% %%     \tableofcontents[currentsection]
  %% %%   \end{frame}

    %% \subsubsection{Agrawal, 94}
    %% % Data structures:
    %% % - LST
    %% % - no SDG
    %% % Algorithm:
    %% % - Intraprocedural
    %% % - On unstructured code
    %%   \begin{frame}
    %%     \frametitle{\subsecname}
    %%     \framesubtitle{\secname}
    %%   \end{frame}
    
    %% \subsubsection{Cifuentes et al., 97} 
    %% % Data structures:
    %% % - no LST
    %% % - no DDG
    %% % - UDC
    %% % Algorithm:
    %% % - Intraprocedural
    %% % - On binary code
    %%   \begin{frame}
    %%     \frametitle{\subsecname}
    %%     \framesubtitle{\secname}
    %%   \end{frame}
    
    %% \subsection{Kiss et al., 03}
    %% % Data structures:
    %% % - SDG specialized to handling binaries
    %% % Algorithm:
    %% % - Intra- and interprocedural
    %% % - On binary code
    %%   \begin{frame}
    %%     \frametitle{\subsecname}
    %%     \framesubtitle{\secname}
    %%   \end{frame}

\end{document}
