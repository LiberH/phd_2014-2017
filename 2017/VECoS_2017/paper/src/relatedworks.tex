\section{Background and related works}
\label{sec:relatedworks}

\subsection{Branch prediction basis}
\label{sec:bpbasis}

In modern processors, pipelines are used to improve the instruction execution rate by executing simultaneously different stages of several instructions at the same time.
Each cycle, one (or more in the case of superscalar processor) instruction is fetched sequentially from the memory and fed into the pipeline.
When a branch instruction is executed, the outcome (whether the branch is taken or not, and what is the actual target) is usually not known in the lower stages of the pipeline.
Thus, bubbles\footnote{A bubble, or pipeline stall, is a delay cycle. When a bubble enters a stage, this stage has no activity during the current cycle.} are inserted in these stages until the address of the next instruction is available. These delays are control hazards and have an impact on the execution time.

Branch prediction is a set of techniques used to minimize the occurrence of this situation.
It consists in trying to predict the outcome of a branch instruction when it gets into the pipeline in order to fetch  the correct following instruction with a high probability.
The simplest form of branch prediction is static branch prediction based on the program code only.
A straightforward prediction algorithm is to predict all branches as always not taken.
If the prediction is correct, no cycle is lost.
If the prediction is incorrect, the lower stages of the pipeline have to be flushed to mimic the insertion of bubbles.
A more efficient algorithm widely used is to predict forward branches (conditional statements) as always not taken and backward branches (loops) as always taken.

Dynamic branch prediction uses runtime information to further improve the prediction accuracy.
There is a wide variety of dynamic branch prediction algorithms that cannot be covered here (see~\cite{Engblom2003} for an overview).
From now on, we will focuses on the algorithm analysed in this paper.
It is based on a set of 2-bit saturating counter as illustrated on figure~\ref{fig:sat_counter}.
From left to right, the four states are usually called: taken (11), weakly taken (10), weakly not taken (01), not taken (00).
A counter is associated with a branch instruction.
It is initialized either statically or after the first execution of the branch.
Then, the state evolves according to the actual outcomes of the execution of the instruction: it is incremented when the branch is taken and decremented otherwise.

\begin{figure}
    \centering
    \begin{adjustbox}{width=\textwidth}
        \begin{tikzpicture}[->,>=stealth',shorten >=1pt,auto,node distance=2.8cm,
    semithick,state/.style=state with output]
  \tikzstyle{every state}=[fill=none,draw=black,text=black]

  \node[state]                                     (WNT) {$01$ \nodepart{lower} nt};
  \node[state]                                     (NT) [right of=WNT] {$00$ \nodepart{lower} nt};
  \node[state]                                     (WT) [left of=WNT] {$10$ \nodepart{lower} t};
  \node[state]                                     (T)  [left of=WT] {$11$ \nodepart{lower} t};

  \path (NT)    edge [bend left]  node {taken} (WNT)
        (NT)    edge [loop right] node {not taken} (NT)
        (WNT)   edge [bend left]  node {not taken} (NT)
        (WNT)   edge [bend left]  node {taken} (WT)
        (WT)    edge [bend left]  node {not taken} (WNT)
        (WT)    edge [bend left]  node {taken} (T)
        (T)     edge [bend left]  node {not taken} (WT)
        (T)    edge [loop left]  node {taken} (T);
\end{tikzpicture}

    \end{adjustbox}
    \caption{State machine of a 2-bit saturating counter. Outputs: \texttt{t} for taken, \texttt{nt} for not taken. The initial state is implementation dependant.}
    \label{fig:sat_counter}
\end{figure}

Each of these counters is usually stored along with the target of the branch instruction in an entry of a small cache memory: the BTB. A BTB entry is commonly retrieved with a combination of index and tag computed from the branch instruction.
When the BTB is full, a replacement policy is applied to free an entry.

\subsection{Analysis of branch prediction techniques}

An important body of work is related to WCET analysis for processor with dynamic branch prediction (see for instance~\cite{Colin2000,Bate2004,Maiza2011,Grund2011625,Puffitsch2016}).
Most of these works focus on the analysis of branch prediction (whether a branch is taken or not) but except for~\cite{Colin2000} and~\cite{Grund2011625}, do not take into account branch target prediction (whether the target of the branch is in the BTB or not).
Only~\cite{Grund2011625} analyzes the interactions between the BTB and the instruction prefetch buffer which is mandatory to take into account fine grain penalty for misprediction.
When this interaction is not analyzed, a uniform penalty must be used to account for misprediction, and the hypothesis of the timing compositionality of the architecture must be implicitly assumed (\textsl{i.e.} the analysis can safely follow local worst case path only~\cite{Wilhelm2009}).  
Lastly none of these works tackle the problem of analyzing the interactions between the ICache and the branch prediction mechanism.
Our approach integrates in a single analysis the ICache, the BTB, and the pipeline using an instruction prefetch buffer. 

\subsection{Model checking and WCET analysis of processors}

There is a limited body of work on model checking techniques for the WCET analysis of processors~\cite{Metzner2004,Gustavsson2010,Dalsgaard2010,Cassez2013}.
In~\cite{Dalsgaard2010} and~\cite{Cassez2013}, it is shown that UPPAAL, a state-of-the-art symbolic model checker for (networks of) timed automata, can be used to model and analyze real-life processors.
The target processor of these works feature instruction and data caches and an in-order 5-stage pipeline without dynamic branch prediction.
Our target processor features an ICache and an in-order 5-stage pipeline with a BTB and an instruction prefetch buffer.
Our ICache model is original, but close to the model used in~\cite{Cassez2013}.
Our pipeline model is fully original, as it integrates an instruction prefetch buffer and interactions with an original BTB model.

Following~\cite{Cassez2013}, to improve scalability, our analysis framework uses program slicing to narrow the set of instructions and memory locations that must be accurately modeled in order to compute a sound bound.
%Contrary to~\cite{Cassez2013}, we do not integrate program slicing and construction of the CFG in a single processing step. % Pas nécessaire.
To improve modularity, we use a standalone, state-of-the-art, program slicer for binary code~\cite{wcet16_mangean}.
