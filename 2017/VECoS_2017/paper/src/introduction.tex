\section{Introduction}

Embedded control systems found in domains like automotive, industrial automation, or robotics, have to satisfy real-time requirements stemming from the dynamics of the physical plant they control.
To design these systems, the worst case execution time (WCET) of the tasks must be computed.
The execution time of a task is a function of its inputs and the initial state of the microarchitecture.
As it is usually not possible to run the real system with all possible combinations of these variables, techniques have been developed to statically estimate upper bounds on the WCET~\cite{Wilhelm2008}.

In this context, different approaches have been investigated. 
One of them is model checking.
%In model checking, an abstract semantics of the system is given in the form of a formal model.
%A specification of the expected behavior is also provided in the form of a temporal logic formula.
%Then the model checker checks the specification with a symbolic exploration of the state space of the abstract system.  
Using model checking in the context of WCET analysis has been debated in the scientific community.
In~\cite{Wilhelm2004} it is deemed as ineffective because of the state space explosion problem.
In~\cite{Metzner2004}, it is claimed that it can actually improve the precision of WCET analysis by leveraging dynamic analysis\footnote{Metzner use dynamic analysis to designate techniques that analyze concrete paths in the system, as opposed to static analysis that consider abstract paths.} of microarchitecture features.
Both points actually hold: model checking allows to compute more precise bounds but suffers from scalability issues.
However, recent results show that model checking sufficiently scales to tackle the WCET analysis of systems based on core such as ARM7 or ARM9~\cite{Dalsgaard2010,Cassez2013}.

In this paper, we investigate the case for model checking in the WCET analysis of architectures
typically found in embedded control systems.
We consider a core architecture inspired by the e200z4 Power 32-bit architecture.
More precisely, we consider a microarchitecture with an instruction cache (ICache), a dynamic branch prediction mechanism, a branch target buffer (BTB), % Faut-il faire la distinction entre les deux ? Idem pour la suite de l'article.
 a prefetch instruction buffer, and a 5-stage pipeline\footnote{The main difference is that the e200z4 is actually a two issues statically scheduled superscalar processor, whereas we consider a single issue processor.}.
The conjoint operation of all these components produce a very complex behaviour that is difficult to analyse with tight and sound static analysis techniques.
We show that model checking techniques can actually be used to compute WCET bounds for this kind of architectures.

\subsubsection{Contribution and outline}
To the best of our knowledge, this paper is the first to propose an analysis integrating at the same time the ICache, the branch target buffer, and the instruction prefetch buffer.
Among the work exploring model checking for WCET analysis, it is the first to tackle a dynamic branch prediction mechanism.
Based on this analysis, we also provide an evaluation of the impact of dynamic branch prediction and BTB on the estimation of WCET for embedded control systems.

The paper is organized as follows. 
In Section~\ref{sec:relatedworks} we provide some background and summarize related works.
In Section~\ref{sec:architecture} we describe our target microarchitecture.
In Section~\ref{sec:overview} we describe our WCET analysis framework.
In Section~\ref{sec:model} we give some insights on the models developed for the dynamic analysis of the target microarchitecture.
In Section~\ref{sec:results} we report an evaluation based on benchmarks. 
In Section~\ref{sec:conclusion} we conclude the paper.

