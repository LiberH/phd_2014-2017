\section{Conclusion}
\label{sec:conclusion}

%Using model checking for WCET analysis has been debated.
In this paper we show that model checking can be used to analyze the complex interactions between the components of a microarchitecture used in safety critical embedded control systems.
We focus on the interaction between the instruction cache, the branch prediction unit, and a pipeline with an instruction buffer.
Model checking provides a solution to perform an integrated analysis of the whole system.
This integrated analysis allows to explore only feasible traces of the system and to compute the actual sequence of memory access requests and pipeline stall states corresponding to each trace.
Our result are promising concerning the scalability of the approach for such systems.

In future works, we shall extend our analysis framework to support programs
that use the stack to store data that impact the control flow. We also want to
produce results using more complex benchmarks, and explore the impact of non-determinism concerning the initial state of the micro-architecture (eg. cache and BTB state).
We will also tend toward having a model aligned with the actual e200z4 core (\textsl{i.e.} adding a second way to the pipeline) in order to validate our model against a real system through microbenchmarks.
Our long term objective is to model and analyze a multiprocessor architecture based on e200z4 core such as the MPC5643L.

